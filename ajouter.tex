\documentclass[11pt,a4paper]{article}
\usepackage[utf8]{inputenc}
\usepackage[francais]{babel}

\usepackage{amsmath}
\usepackage{amsfonts}
\usepackage{amssymb}
\usepackage{graphicx}
\usepackage[left=2cm,right=2cm,top=2cm,bottom=2cm]{geometry}

\title{Écoulements biologiques}
\author{Tiago Lobato Gimenes \\ Chongmo Liu}

\numberwithin{equation}{subsection}
\numberwithin{figure}{subsection}

\begin{document}
\setlength{\parindent}{0pt}
\maketitle

%
%		INTRODUCTION
%
\section*{Introduction}

TODO: ecrire introduction

%
%		SECTION
%
\section{Problème proposé}

Le problème proposé a été d'étudier les écoulements biologiques, plus précisément, l'écoulement sanguin dans des petites artères.
 
Le modèle mathématique proposé pour résoudre ce problème a eu une évolution au cours du modal. Premièrement nous avons commencé avec un modèle simple d'écoulement d'un fluide incompressible dans un tube avec des parois fixes. Nous avons aussi supposé que le nombre de Reynolds de  l'écoulement est petit et que les forces volumiques sont négligeables, ce que nous a permit d'utiliser le modèle suivante:
\begin{equation}
\begin{aligned}
& -\nu \Delta u + \nabla p = 0 \\
& -\mathrm{div}\;u = 0 \label{Stokes}
\end{aligned}
\end{equation}

\section{Le cas du rectangle 2D}
Tout d'abord, on commence par une simulation qui nous permet de trouver une solution du problème \ref{Stokes} dans un rectangle $\Omega = \{(x,y) \in \mathbb{R}^2 \;t.q\; 0 \leq x \leq L, \; -R/2 \leq y \leq R/2\}$. De plus, on fait certaines hypothèses de la vitesse:\\
Elle est nulle le long de deux bords horizontaux.\\
Elle est parabolique sur le bord $x=0$: vitesse nulle aux parois et maximale à mi-hauteur.\\
Elle suit une condition de Neumann sur le bord $x=L$.\\
Ainsi, le problème que l’on considère avec des conditions aux limites est suivant :\\

\begin{equation}
\begin{aligned}
& \underline{u}(x,-R) = \underline{0} \\
& \underline{u}(x, R) = \underline{0} \\
& \underline{u}(0,y) = \left(R^2 - y^2, 0\right) \\
& \nu\frac{\partial\underline{u}}{\partial n}(L,y) = p.\underline{n}\\
\end{aligned} \label{LimitesClassiques}
\end{equation}

Soitent les deux espaces des fonctions tests\\
$V=\left\{\underline{v}\in H^1(\Omega)^2, \quad  \underline{u}(x,-R)=\underline{u}(x, R)=\underline{u}(0,y)=0\right\} \quad et \quad Q=\left\{q\in L^2(\Omega)\right\}$\\
La formulation variationnelle du problème \ref{Stokes} avec les conditions \ref{LimitesClassiques} s’écrit :\\
Trouver $\underline{u}\in V $ et $p \in Q $ tels que\\

\begin{equation}
\begin{aligned}
\nu \int\limits_\Omega \nabla \underline{u} \nabla \underline{v} \;\mathrm{d}\Omega - \int\limits_\Omega p \mathrm{div}  \underline{v} \; \mathrm{d}\Omega = 0 \quad \forall \underline{v}\in V\\
\int\limits_\Omega q \mathrm{div} \underline{u} \;\mathrm{d}\Omega = 0 \quad \forall q \in Q\\
\end{aligned}\label{FormuleVariationnelleSansTranspose}
\end{equation}

\subsection{Simulation de la formulation variationnelle \ref{FormuleVariationnelleSansTranspose}}

En faisant la simulation de la formulation variationnelle \ref{FormuleVariationnelleSansTranspose} avec les conditions aux limites \ref{LimitesClassiques} nous avons comme résultat les figures \ref{StokesConditionsClassiques} et \ref{StokesCondtiionsClassiquesPression}, ce que correspond bien à l'écoulement de Poiseuille, solution analytique dans ce cas.



Nous avons aussi fait une maillage non régulière qui est plus dense dans les endroit où la pression varie le plus. Les résultats peuvent êtres vus dans les figures \ref{StokesConditionsClassiquesVitessesIrregulier} et \ref{StokesConditionsClassiquesPressionIrregulier}.


\subsection{Simulation de la formulation variationnelle \ref{FormuleVariationnelleAvecTranspose}}

Vu que $\mathrm{div} u = 0$, La formulation variationnelle \ref{FormuleVariationnelleSansTranspose} peut avoir une autre forme \ref{FormuleVariationnelleAvecTranspose} qui peret la la même solution. 
\begin{equation}
\begin{aligned}
\nu \int\limits_\Omega \left(\nabla \underline{u} + \nabla \underline{u}^t\right) \nabla \underline{v} \;\mathrm{d}\Omega - \int\limits_\Omega p\mathrm{div}\underline{v} \; \mathrm{d}\Omega = 0\quad \forall \underline{v}\in V\\
\int\limits_\Omega q \mathrm{div} \underline{u} \;\mathrm{d}\Omega = 0 \quad \forall q \in Q\\
\label{FormuleVariationnelleAvecTranspose}
\end{aligned}
\end{equation}

Prenons $v$, la même fonction que nous avons utilisé ci-dessus. En intégrant par parties et en développant le calcul, nous arrivons aux conditions aux bords.
\begin{equation}
\begin{aligned}
& \underline{u}(x,-R) = \underline{0} \\
& \underline{u}(x, R) = \underline{0} \\
& \underline{u}(0,y) = \left(R^2 - y^2, 0\right) \\
& \nu\left(\frac{\partial\underline{u}}{\partial n}(L,y) + \frac{\partial\underline{u}}{\partial n}^t(L,y)\right) = p.\underline{n}
\end{aligned} \label{Limites}
\end{equation}

En simulant la formule variationnelle \ref{FormuleVariationnelleAvecTranspose} avec les conditions aux bords \ref{Limites} nous avons les figures \ref{StokesLimitesTransposeVitesses} et \ref{StokesLimitesTransposePression}.


Nous avons aussi fait une maillage non régulière qui est plus dense dans les endroit où la pression varie le plus. Les résultats peuvent êtres vus dans les figures \ref{StokesLimitesTransposeVitessesIrregulier} et \ref{StokesLimitesTransposePressionIrregulier}.

%
%		SECTION
%
\section{Le cas du cylindre 3D}

Pour résoudre le cas tridimensionnel nous suppose qu'il y a une symétrie axial, ce que nous rendre possible la décomposition du problème en 3D en 2D. Avec notre supposition de symétrie coaxiale, les formulations variationnelles \ref{FormuleVariationnelleAvecTranspose} et \ref{FormuleVariationnelleSansTranspose} deviennent, en coordonnées polaires, les formulations \ref{FormuleVariationnelleAvecTransposeCylindrique} et \ref{FormuleVariationnelleSansTransposeCylindrique} respectivement.
\begin{equation}
\begin{aligned}
& 2\pi\int\limits_0^r\int\limits_0^L \left(\frac{\partial u_r}{\partial r}\frac{\partial v_r}{\partial r} + \frac{\partial u_r}{\partial z}\frac{\partial v_r}{\partial z} + \frac{u_rv_r}{r^2} + \frac{\partial u_z}{\partial r}\frac{\partial v_z}{\partial r} + \frac{\partial u_z}{\partial z}\frac{\partial v_z}{\partial z}\right)r\mathrm{drdz}  \quad- \\
& 2\pi\int\limits_0^r\int\limits_0^L \left(p\left[\frac{\partial v_r}{\partial r} + \frac{v_r}{r} + \frac{\partial v_z}{\partial z}\right] - q\left[\frac{\partial u_r}{\partial r} + \frac{u_r}{r} + \frac{\partial u_z}{\partial z}\right]\right) r\mathrm{drdz} = 0
\end{aligned} \label{FormuleVariationnelleSansTransposeCylindrique}
\end{equation}

\begin{equation}
\begin{aligned}
& 2\pi\int\limits_0^r\int\limits_0^L \left(\frac{\partial u_r}{\partial r}\frac{\partial v_r}{\partial r} + \frac{\partial u_r}{\partial z}\frac{\partial v_r}{\partial z} + \frac{u_rv_r}{r^2} + \frac{\partial u_z}{\partial r}\frac{\partial v_z}{\partial r} + \frac{\partial u_z}{\partial z}\frac{\partial v_z}{\partial z}\right)r\mathrm{drdz}  \quad+ \\
& 2\pi\int\limits_0^r\int\limits_0^L \left(\frac{\partial u_r}{\partial r}\frac{\partial v_r}{\partial r} + \frac{\partial u_z}{\partial r}\frac{\partial v_r}{\partial z} + \frac{u_rv_r}{r^2} + \frac{\partial u_r}{\partial z}\frac{\partial v_z}{\partial r} + \frac{\partial u_z}{\partial z}\frac{\partial v_z}{\partial z}\right)r\mathrm{drdz}  \quad+ \\
& 2\pi\int\limits_0^r\int\limits_0^L \left(p\left[\frac{\partial v_r}{\partial r} + \frac{v_r}{r} + \frac{\partial v_z}{\partial z}\right] - q\left[\frac{\partial u_r}{\partial r} + \frac{u_r}{r} + \frac{\partial u_z}{\partial z}\right]\right) r\mathrm{drdz} = 0
\end{aligned} \label{FormuleVariationnelleAvecTransposeCylindrique}
\end{equation}

où $\underline{u} = u_r\underline{e}_r + u_z\underline{e}_z$ et $\underline{v} = v_r\underline{e}_r + v_z\underline{e}_z$. Pour simuler les formules variationnelles \ref{FormuleVariationnelleAvecTransposeCylindrique} et \ref{FormuleVariationnelleSansTransposeCylindrique} nous avons utilisée la symétrie coaxiale pour simuler juste la moitié du cylindre, ce que nous donne les conditions \ref{ConditionsMoitieTubeAvecTranspose} et \ref{ConditionsMoitieTubeSansTranspose} pour les formules variationnelles \ref{FormuleVariationnelleAvecTransposeCylindrique} et \ref{FormuleVariationnelleSansTransposeCylindrique} respectivement.
\begin{equation}
\begin{aligned}
& \underline{u}(x,-R) = \underline{0} \\
& \nu\frac{\partial\underline{u}}{\partial n}(x,0) = p.\underline{n} \\
& \underline{u}(0,y) = \left(R^2 - y^2, 0\right) \\
& \nu\frac{\partial\underline{u}}{\partial n}(L,y) = p.\underline{n}
\end{aligned} \label{ConditionsMoitieTubeSansTranspose}
\end{equation}

\begin{equation}
\begin{aligned}
& \underline{u}(x,-R) = \underline{0} \\
& \nu\left(\frac{\partial\underline{u}}{\partial n}(x,0) + \frac{\partial\underline{u}}{\partial n}^t(x,0)\right) = p.\underline{n} \\
& \underline{u}(0,y) = \left(R^2 - y^2, 0\right) \\
& \nu\left(\frac{\partial\underline{u}}{\partial n}(L,y) + \frac{\partial\underline{u}}{\partial n}^t(L,y)\right) = p.\underline{n}
\end{aligned} \label{ConditionsMoitieTubeAvecTranspose}
\end{equation}

\subsection{Simulation de la formule variationnelle \ref{FormuleVariationnelleSansTransposeCylindrique}}

En utilisant la formule variationnelle \ref{FormuleVariationnelleSansTransposeCylindrique} et les conditions aux bords \ref{ConditionsMoitieTubeSansTranspose} nous arrivons au figures \ref{StokesClassiqueVitessesCylindrique} et \ref{StokesClassiquePressionCylindrique}.


Nous avons aussi fait une maillage non régulière qui est plus dense dans les endroit où la pression varie le plus. Les résultats peuvent êtres vus dans les figures \ref{StokesClassiqueVitessesCylindriqueIrregulier} et \ref{StokesClassiquePressionCylindriqueIrregulier}



\subsection{Simulation de la formule variationnelle \ref{FormuleVariationnelleAvecTransposeCylindrique}}

En utilisant la formule variationnelle \ref{FormuleVariationnelleAvecTransposeCylindrique} et les conditions aux bords \ref{ConditionsMoitieTubeAvecTranspose} nous arrivons au figures \ref{StokesLimitesVitesseCylindrique} et \ref{StokesLimitesPressionCylindrique}


Nous avons aussi fait une maillage non régulière qui est plus dense dans les endroit où la pression varie le plus. Les résultats peuvent êtres vus dans les figures \ref{StokesLimitesVitessesCylindriqueIrregulier} et \ref{StokesLimitesPressionCylindriqueIrregulier}


%
%	SECTION
%
\section{Le cas 3D d'une forme symétrique non cylindrique}

Il a été proposé de faire l'étude d'un écoulement dans un objet qui avait toujours une symétrie coaxiale mais qui n'avait plus la forme d'un cylindre. 

\subsection{L'étude d'un cylindre avec une ellipse de révolution}

Une forme choisit pour faire cet étude a été un cylindre qui avait une ellipse de révolution. En utilisant la formule variationnelle \ref{FormuleVariationnelleSansTransposeCylindrique} et les conditions aux limites \ref{ConditionsMoitieTubeSansTranspose} nous avons les figures \ref{StokesClassiqueVitessesCylindreEllipse} et \ref{StokesClassiquePressionCylindreEllipse}


Nous avons aussi fait une maillage non régulière qui est plus dense dans les endroit où la pression varie le plus. Les résultats peuvent êtres vus dans les figures \ref{StokesClassiqueVitessesCylindreEllipseIrregulier} et \ref{StokesClassiquePressionCylindreEllipseIrregulier}

\clearpage


\end{document}